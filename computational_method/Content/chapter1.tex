\chapter{插值}
插值法往往是用一组多项式来表示一个满足特定条件的函数 $f(x)$,其中特殊条件往往是函数 $f(x)$ 或其导数在某点的值. 另外,{\color{red} 本指南的求和部分默认为 $0$ 到 $n-1$ 共 $n$ 个数。}

\section{Lagrange 插值}
我们考虑一组多项式作为函数 $f(x)$ 的基:
\[
    l_i(x) = \dfrac{\prod_{k\neq i}(x-x_k)}{\prod_{k\neq i}(x_i-x_k)}.
\]
若我们知道函数 $f(x)$ 的 $n$ 个点 $f(x_m)$ 的函数值 $f(x_m)$,我们可以得到一个拟合多项式:
\[
    L_n(x) = \sum_{m} l_i(x_m)f(x_m).
\]
我们称 $L_n(x)$ 为 {\bf Lagrange 多项式}. 考虑 $l_i$ 的性质 $l_i(x_j) = \delta_{ij}$,显然有 $L_n(x_m) = f(x_m)$.

对于一个插值多项式,我们更关注多项式的误差. 可以证明,Lagrange 多项式的误差为:
\[
    R_n(x) = \dfrac{f^{(n+1)}(\xi)}{(n+1)!}\prod_i( x-x_i),\quad \xi \in [a,b].
\]
\begin{problembox}
    \begin{example}
        试证明 Lagrange 多项式的误差可以写为上述的 $R_n(x)$.
    \end{example}
\end{problembox}

\section{Newton 插值}

