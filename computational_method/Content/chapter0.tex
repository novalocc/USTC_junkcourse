\chapter*{绪论与数学准备}
\addcontentsline{toc}{chapter}{绪论与数学准备}

正如前言所说计算方法作为一门{\bf 数学课},主要讲述了计算机进行数学计算的方法——由于计算机只能处理离散的数字信号,因此对于微积分、函数拟合等操作,我们必须将其转化为计算机能够进行的数学操作(主基本的四则运算)来完成。同时还要考虑算法的普适性、收敛性、稳定性和时间复杂度等一系列问题。

在开始介绍各种算法之前,我们先来回顾一些在微积分与线性代数\footnote{广义上的,实际上很多概念并不会在“线性代数”这门课程中出现.}的定义与定理。值得一提的是,尽管这是一门数学课,但我们在一些表述上并不会追求十分的严谨,例如我们似乎总是默认矩阵的维数是“恰当”的(方阵或与所作用的向量相匹配),本讲义中将不过多赘述。

\section{微分中值定理}
常见的三个微分中值定理:
\begin{theorem}{微分中值定理}{thm:0.1}
设 $f(x)$ 在区间 $[a,b]$ 上满足相应条件,则:
\begin{enumerate}
    \item \textbf{罗尔定理}:若 $f(x) \in C^1[a,b]$,且 $f(a) = f(b)$,则存在 $\xi \in (a,b)$,使得
    \[
        f'(\xi) = 0.
    \]

    \item \textbf{拉格朗日中值定理}:若 $f(x) \in C^1[a,b]$,则存在 $\xi \in (a,b)$,使得
    \[
        f'(\xi) = \frac{f(b) - f(a)}{b - a}.
    \]

    \item \textbf{泰勒定理}:若 $f \in C^{n+1}[a,x]$,则有
    \[
        f(x) = \sum_{k=0}^{n} \frac{f^{(k)}(a)}{k!}(x - a)^k + R_n(x),
    \]
    其中余项 $R_n(x)$ 有两种形式:

    \begin{itemize}
        \item 拉格朗日余项:
        \[
            R_n(x) = \frac{f^{(n+1)}(\xi)}{(n+1)!}(x - a)^{n+1}, \quad \xi \in (a,x)
        \]
        
        \item 积分余项:
        \[
            R_n(x) = \frac{1}{n!} \int_a^x (x - t)^n f^{(n+1)}(t) \, dt
        \]
    \end{itemize}

    若 $|f^{(n+1)}(t)| \leqslant M$,则误差估计为:
    \[
        |R_n(x)| \leqslant \frac{M}{(n+1)!} |x - a|^{n+1}.
    \]
\end{enumerate}
\end{theorem}


\section{误差}
在计算机计算中我们得到的数往往是一个近似的结果,因此为了衡量结果的准确性我们需要定义数据的误差。
\begin{definition}{误差}{def:0.1}
    对于待求数据的精确值 $x^\star$ 和计算出的数据的值 $x$,我们定义{\bf (绝对)误差}|:
    \[
        e = x^\star - x.
    \]
    若要求 $|e|<\varepsilon$,我们称 $\varepsilon$ 为{\bf 误差限};同时定义更为朴实的相对误差 $\delta = \dfrac{e}{x^\star}$.
\end{definition}

\section{范数}
\begin{definition}{范数}{def:0.2}
    在实数域 $\mathbb{R}$ 上的线性空间 $V$ 中,我们定义映射 $\| \cdot \| :V\to \mathbb{R}$,满足:
    \begin{enumerate}
        \item 非负性:$\|\alpha\| \geqslant 0,\; \alpha \in V$,当且仅当 $\alpha =0$ 时等号成立;
        \item 齐次性:$\|\lambda \alpha\| =|\lambda|\|\alpha \|,\; \lambda \in \mathbb{R},\, \alpha \in V$;
        \item 三角不等式:$\|\alpha+\beta\|\leqslant \|\alpha\|+\|\beta\|,\; \alpha,\beta \in V$.
    \end{enumerate}
\end{definition}

常见的范数有:
\begin{enumerate}
    \item (向量)$p$-范数:$\|\alpha\|_p = \left(\sum_{i=1}^n |\alpha_i|^p\right)^{\frac{1}{p}},\; p\geqslant 1$;
    \item (矩阵)$p$-范数:$\|\mathbf{A}\|_p = \max\dfrac{\|\mathbf{A}\alpha\|_p}{\|\alpha\|_p},\; p\geqslant 1$,我们往往将 $\alpha$ 限制在单位球面上.
    \item 特殊的 $p$-范数:$\|\alpha\|_{\infty} = \max |\alpha_i|$ 称为向量范数,$\|\mathbf{A}\|_1 $ 为列和范数,$\|\mathbf{A}\|_\infty$ 为行和范数;
    \item Frobenius 范数:$\|\mathbf{A}\|_F = \sqrt{\tr(\mathbf{A}^\top\mathbf{A})} = \sqrt{\left|\sum_{i,j} a_{ij}\right|^2}$;
    \item 条件数:$\cond_p (\mathbf{A} ) = \|\mathbf{A}\|_p \|\mathbf{A}^{-1}\|_p$,$\cond_p (\mathbf{A}) \geqslant \|\mathds{1}\|_p= 1$,因此条件数约大稳定性越差.
\end{enumerate}

\section{谱半径}
\begin{definition}{谱半径}{def:0.3}
    矩阵 $\mathbf{A}$ 的{\bf 谱半径}定义为:
    \[
        \rho(\mathbf{A}) = \max\{|\lambda_i|\}.
    \]
    其中 $\lambda_i$ 是 $\mathbf{A}$ 的第 $i$ 个本征值\footnote{eigenvalue,也称特征值,对于 eigen- 前缀,本讲义统称本征.}.
\end{definition}
对于任意矩阵 $\mathbf{A}$ ,我们有:
\[
    \rho(\mathbf{A}) \leqslant \|\mathbf{A}\|_p,\quad \forall p.
\]
且有恒等式:
\[
    \|  \mathbf{A}\|_2 = \sqrt{\rho(\mathbf{A}^\top\mathbf{A})}.
\]

同时我们可以定义{\bf 矩阵的收敛性},即一下三个命题相互等价:
\begin{enumerate}
    \item 矩阵 $\mathbf{A}$ 收敛;
    \item $\lim_{k\to \infty} \mathbf{A}^k = 0$;
    \item $\rho(\mathbf{A})<1$.
\end{enumerate}
