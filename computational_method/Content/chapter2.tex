\chapter{数值微积分}
\section{数值微分}
要求解函数 $f(x)$ 在 $x$ 点的数值微分,我们往往选取步长 $h$,用差商代替微分:
\[
    f'_1(x) = \dfrac{f(x+h)-f(x)}{h},\quad f'_2(x) = \dfrac{f(x)-f(x-h){h}},\quad f'_3(x) = \dfrac{f(x+h)-f(x-h)}{2h}.
\]
三种差商分别被称为向前差商、向后差商和中心差商。利用泰勒展开可以证明三种差商的误差分别为:
\[
    R_1(x) = R_2(x) = \dfrac{h}{2!}f'(\xi) \sim O(h),\quad R_3 (x) = -\dfrac{h^2}{3!}f''(\xi)\sim O(h^2).
\]

\section{数值积分}
在介绍数值积分的方法之前,我们先介绍一种重要的衡量指标:{\bf 代数精度}.
\begin{definition}{代数精度}
    记函数 $f$ 的数值积分和积分的精确解分别为 $I_n(f)$ 和 $I(f)$,若数值解分满足:
    \[
        I_n(x^i) = I(x^i),\quad i = 0,1,2,\dots k,\qquad I_n(x^{k+1})\neq I(x^{k+1}).
    \]
    则称数值积分 $I_n$ 具有 $k$ 阶代数精度。
\end{definition}
代数精度实际上是利用多项式衡量数值积分准确定的一种方式,具有 $k$ 阶代数精度数值积分 $I_n$ 在求解不大于 $k$ 阶的多项式的积分时最后的结果是准确的.

现在回到求解数值积分的方法,我们往往利用离散的求和代替积分:
\[
    I_n(f) = \sum_i a_if(x_i).
\]