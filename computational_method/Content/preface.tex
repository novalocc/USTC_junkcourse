\chapter*{前言}
\addcontentsline{toc}{chapter}{前言}
计算方法作为物理学专业必修课(天文、少物选修),其恶心程度甚至可以与计算物理并肩:编程作业从零开始造轮子,书面作业堪称计算机化学生\footnote{该梗来自“摩托化步兵”.}(也称手算方法),到了最后考试却成为了证明大赛,真是让学生叫苦不迭。

实际上计算方法这门课的平时作业利用 AI 可以很不错的完成(笔者使用 GPT 4o 模型),因此真正的重头戏在最后的期末考试。

在编写这份讲义时,笔者正在从零开始复习计算方法,不难通过版本号知道,本篇讲义于 6.25 开始编写,而计算方法期末在 6.27. 为了帮助其余学生两天从零速通计算方法期末考试,我整理了刘元彻同学的笔记,并附带历年真题与题目讲解,制作了这份指南,希望能对更多的被计算方法折磨的学生产生帮助,远离期末考试之苦。

本讲义主要参考刘元彻同学的手写笔记。感谢应跃洲同学牵头发起本项目,并邀请我协助完成,感谢刘元彻学长与施耀炜学长的支持。祝大家食用愉快!


\begin{itemize}
    \item 2025.6.25 本资料由刘元彻同学手写笔记整理而来,目前版本号为v0.0.1。
\end{itemize}