\chapter{积分变换法和格林函数}
本章主要针对的是无界空间中的方程求解。
\section{积分变换}
这里会用到的变换在数学分析和复变函数课中都应该了解过。
\subsection{傅里叶变换}
如果求解的区域是无界区域,我们经常考虑傅里叶变换。不同教材对傅里叶变换的定义可能会不同。这门课采用的定义是
\[F[f] = F(\lambda) = \int_{-\infty}^{+\infty} f(x) \ee^{-\ii\lambda x} \dd{x}\]
对应的逆变换为
\[f(x) = \frac{1}{2 \pi} \int_{-\infty}^{+\infty} F(\lambda) \ee^{\ii \lambda x} \dd{\lambda}\]

下面罗列出了傅里叶变换的性质,我们不关心这些性质是怎么来的,记住就好。
\begin{itemize}
    \item 线性性:$F[C_1 f + C_2 g] = C_1 F[f] + C_2 F[g]$
    \item 频移性:$F[f(x) \ee^{\ii \lambda_0 x}] = F(\lambda - \lambda_0)$
    \item 位移性:$F[f(x + a)] = F(\lambda) \ee^{\ii \lambda a}$
    \item 相似性:$a > 0, \, F[f(ax)] = \frac{1}{a} F(\frac{\lambda}{a})$
    \item \textcolor{red}{微分性质}\footnote{这条性质非常重要,它把求导运算化为了乘积运算,从而简化了方程的求解。}: $F[f^{(n)}] = (\ii \lambda)^n F[\lambda]$
    \item \textcolor{red}{卷积性质}:$F[f * g] = F[f] \times F[g]$。这里卷积定义为
        \[(f * g)(x) = \int_{-\infty}^{+\infty} f(s) g(x - s) \dd{s}\]
        这个性质经常用来求解逆变换
        \[F^{-1}[F G] = F^{-1}[F] * F^{-1}[G]\]
\end{itemize}

还需要记住的是两个及其重要的公式
\color{red}
\[F[\ee^{- a x^2}] = \sqrt{\frac{\pi}{a}} \ee^{-\frac{\lambda^2}{4a}},\, F^{-1}[\ee^{a \lambda^2}] = \frac{1}{\sqrt{4 \pi a}} \ee^{-\frac{x^2}{4a}}\]
\color{black}
本质上这就是高斯积分
\[\int_{0}^{+\infty} \ee^{-a x^2} \dd{x} = \sqrt{\frac{\pi}{a}}\]

我们也会碰到高维的傅里叶变换,实际上我们只会关心微分性质和卷积性质,这其实只需要把导数换成偏导数,然后把卷积从一重积分变成$n$重积分就可以了。
\begin{problembox}
    \begin{example}
        求解热传导方程的初值问题
        \begin{equation*}
            \left\{
                \begin{aligned}
                    &u_t = u_{xx} + u,\, -\infty < x < +\infty,\, 0 \leq t < +\infty,\\
                    &u(0, x) = \ee^{-x^2},\, -\infty < x < +\infty
                \end{aligned}
            \right.
        \end{equation*}
    \end{example}
    \begin{solution}
        做傅里叶变换。记
        \[\bar{u}(t, \lambda) = \int_{-\infty}^{+\infty} u(t, x) \ee^{- \ii \lambda x} \dd{x}\]
        根据微分性质,得到
        \[\bar{u}_t = -\lambda^2 \bar{u} + \bar{u} \Rightarrow \bar{u}(t, \lambda) = C(\lambda) \ee^{-\left(\lambda^2 - 1\right)t}\]
        由初始条件,得到
        \[\bar{u}(0, \lambda) = F[\ee^{-x^2}] \Rightarrow C(\lambda) = F[\ee^{-x^2}]\]
        所以逆变换回去就可以了。要注意逆变换是对$\lambda$做变换
        \[u(t, x) = F^{-1}\left[F[\ee^{-x^2}]\right] * F^{-1}[\ee^{-\left(\lambda^2 - 1\right)t}] = \ee^{-x^2} * \left(\frac{\ee^t}{\sqrt{4 \pi t}} \ee^{\frac{x^2}{4t}}\right) = \frac{\ee^{t}}{\sqrt{4 \pi t}}\int_{-\infty}^{+\infty} \ee^{-\left(x - s\right)^2} \ee^{-\frac{s^2}{4t}}\]
        配一下方,然后用高斯积分公式,得到结果为
        \[u(t, x) = \frac{1}{\sqrt{1 + 4t}} \ee^{t - \frac{x^2}{1 + 4t}}\]
    \end{solution}
\end{problembox}
\begin{problembox}
    \begin{exercise}
        用傅里叶变换推导达朗贝尔公式。
    \end{exercise}
\end{problembox}

\subsection{拉普拉斯变换}
拉普拉斯变换用的不是很多,如果要解决的是$0 \leq t < +\infty$的初值问题,我们或许会考虑它。我见过的唯一的例子就是求解半无限长弦问题。

拉普拉斯变换的定义是
\[L[f] = L(p) = \int_{0}^{+\infty} f(x) \ee^{-p t} \dd{t}\]
这里的$p$实际上是定义在上半复平面上的复数,但实际用起来没人关心。逆变换是
\[f(t) = \frac{1}{2 \pi \ii} \int_{-\infty}^{+\infty} L(\sigma + \ii \lambda) \ee^{\ii \lambda t} \dd{\lambda}\]
如果我们还记得留数定理,会发现上面的式子也可以写成
\[f(t) = \sum_i \Res[L(p) \ee^{p t}, p_i]\]

下面罗列出了拉普拉斯变换的性质,我们也只需要记住。
\begin{itemize}
    \item 线性性:$L[C_1 f + C_2 g] = C_1 L[f] + C_2 [g]$
    \item 频移性:$L[f(t) \ee^{\lambda t}] = L(p - \lambda)$
    \item \textcolor{red}{延迟性}\footnote{这个性质和卷积性质一起常用来求解逆变换,见下面的例子。}:$\tau > 0,\, L[f(t - \tau) h(t - \tau)] = L(p) \ee^{- p \tau}$
    \item \textcolor{red}{微分性质}:$L[f^{\left(n\right)}(t)] = p^n L\left(p\right) - p^{n-1} f\left(0+\right) - p^{n-2} f^{\left(1\right)}\left(0+\right) - \cdots$
    \item 积分性质:$L[\int_{0}^{t} f(s) \dd{s}] = \frac{L(p)}{p}$
    \item \textcolor{red}{卷积性质}:$L[f * g] = L[f] \times L[g]$。这里卷积定义为
        \[(f * g)(t) = \int_{0}^{t} f(s) g(t-s) \dd{s}\]
        同样的,更常用的是逆变换
        \[L^{-1}[F G] = L^{-1}[F] * L^{-1}[G]\]
\end{itemize}

要记住的还有下面常见的拉普拉斯变换:
\color{red}
\[L[\ee^{\lambda t}] = \frac{1}{p - \lambda},\, L[\frac{t^n}{n!}] = \frac{1}{p^{n+1}},\,L[\sin(\omega t)] = \frac{\omega}{p^2 + \omega^2},\, L[\cos(\omega t)] = \frac{p}{p^2 + \omega^2}\]
\color{black}
\begin{problembox}
\begin{example}
    解混合问题:
    \begin{equation*}
        \left\{
    \begin{aligned}
        &u_{tt} = a^2 u_{xx} + f\left(t\right),\, x>0, \,0 \leq t<+\infty\\
        &u(0, x) = 0, u_t(0, x) = 0\\
        &u(t, 0) = 0, u(t, +\infty)\,\text{有界}  
    \end{aligned}
    \right.
    \end{equation*}
\end{example}
\begin{solution}
做拉普拉斯变换。设
\[U(p, x) = \int_{0}^{+\infty} u(t, x) \ee^{- p t} \dd{t}\]
根据微分关系和初始条件得到
\[p^2 U(p, x) = a^2 U_{xx}(p, x) + L[f(t)]\]
所以先求齐次方程的通解,这一眼就瞪出来了
\[W(p, x) = C_1(p) \ee^{\frac{px}{a}} + C_2(p) \ee^{-\frac{px}{a}}\]
然后找一个最简单的特解。注意到$L[f(t)]$只和$p$有关,所以很容易猜出来一个特解
\[V(p, x) = \frac{L[f(t)]}{p^2}\]
并且,因为$u(t, +\infty)$有限,所以$U(p, +\infty)$也应该有限,这样就得到
\[U(p, x) = \frac{L[f(t)]}{p^2} + C_2(p) \ee^{-\frac{px}{a}}\]
令$x = 0$,得到$C_2(p) = -\frac{L[f(t)]}{p^2}$。于是
\[U(p, x) = \frac{L[f(t)]}{p^2} - \frac{L[f(t)]}{p^2} \ee^{-\frac{p x}{a}}\]
接下来无非是把上面那一堆东西逆变换回去。先做第一项:
\[L^{-1}\left[\frac{L[f(t)]}{p^2}\right] = L^{-1}[L(t)] * L^{-1}[\frac{1}{p^2}] = \int_{0}^{t} f(t - s) s \dd{s}\]
第二项和第一项相比,多乘了一个$\ee^{-p \frac{x}{a}}$。根据延迟性质得到
\[L^{-a}\left[-\frac{L[f(t)]}{p^2} \ee^{-p \frac{x}{a}}\right] = -\left(\int_{0}^{t - \frac{x}{a}} f\left(t - \frac{x}{a} - s\right) s \dd{s}\right) \times h(t - \frac{x}{a})\]
这两项和起来就是答案了。整理一下,最终的解为
\begin{equation*}
    u(t, x) = 
    \left\{
        \begin{aligned}
            &\int_{0}^{t} f(t - s) s \dd{s},\, t < \frac{x}{a}\\
            &\int_{0}^{t} f(t - s) s \dd{s} - \int_{0}^{t - \frac{x}{a}} f\left(t - \frac{x}{a} - s\right) s \dd{s},\, t \geq \frac{x}{a}
        \end{aligned}
    \right.
\end{equation*}
\end{solution}
\end{problembox}
\begin{problembox}
    \begin{exercise}
        解定解问题:
        \begin{equation*}
            \left\{
                \begin{aligned}
                    &u_{tt} = a^2 u_{xx},\, 0 < x < l,\, 0 \leq t < +\infty,\\
                    &u(t, 0) = 0,\, u_x(t, l) = A \sin(\omega t),\\
                    &u(0, x) = 0,\, u_t\left(0, x\right) = 0.
                \end{aligned}
            \right.
        \end{equation*}
    \end{exercise}
    \begin{exercise}
        解定解问题:
        \begin{equation*}
            \left\{
                \begin{aligned}
                    &u_{t} = a^2 u_{xx},\, x > l,\, 0 \leq t < +\infty,\\
                    &u(t, 0) = f(t),\, u(t, +\infty)\,\text{有界}\\
                    &u(0, x) = 0.
                \end{aligned}
            \right.
        \end{equation*}
    \end{exercise}
\end{problembox}

\section{基本解和格林函数}
\subsection{基本解}
在了解基本解之前,我们需要先回顾一下$\delta$函数的概念。物理上引入$\delta$函数,可以考虑质点的密度分布,或者点电荷的电荷密度分布。
但对于做题而言,只需要记住$\delta$函数的以下性质就好:
\begin{itemize}
    \item 定义:
        \begin{equation*}
            \delta(x) = 
            \left\{
                \begin{aligned}
                    &0, x \neq 0,\\
                    &+\infty, x = 0
                \end{aligned}
            \right.
        \end{equation*}
        \[\int_{-\infty}^{+\infty} \delta(x) \dd{x} = 1\]
    \item 筛选性质
        \[{\color{red}{\int_{-\infty}^{+\infty} \delta(x - x_0) f(x) \dd{x} = f(x_0)}}\]
    \item 傅里叶变换
        \[F[\delta(x)] = 1,\, F^{-1}[1] = \delta(x)\]
        实际上就是记住下面的等式\footnote{因为$\sin x$是奇函数,所以如果用欧拉公式展开$\ee^{\ii x}$并且对$x$积分,$\sin x$的结果一定是0。}:
        \[{\color{red}\frac{1}{2 \pi} \int_{-\infty}^{+\infty} \ee^{\ii \lambda \left(x - x_0\right)} \dd{\lambda} = \frac{1}{2\pi} \int_{-\infty}^{+\infty} \cos(\lambda(x - x_0)) \dd{\lambda} = \delta(x - x_0)}\]
\end{itemize}
我们也会遇到$n$维的$\delta$函数,实际上就是$n$个一维$\delta$函数的乘积:
\[\delta^{\left(n\right)}(x_1, \dots, x_n) = \prod_{i = 1}^{n}\delta(x_i)\]
所以$n$维$\delta$函数具有和一维$\delta$函数完全类似的性质。
\begin{problembox}
    \begin{example}\label{eg:3.3}
        解初始问题
        \begin{equation*}
            \left\{
                \begin{aligned}
                    &u_{tt} = a^2 u_{xx}, 0 \leq t < +\infty,\, -\infty < x < +\infty,\\
                    &u(0, x) = \delta(x - \xi),\, u_t(0, x) = 0
                \end{aligned}
            \right.
        \end{equation*}
    \end{example}
    \begin{solution}
        做傅里叶变换。记
        \[U(t, \lambda) = \int_{-\infty}^{+\infty} u(t, x) \ee^{-\ii \lambda x} \dd{x}\]
        则有
        \begin{equation*}
            \left\{
                \begin{aligned}
                    &U_{tt} = -a^2 \lambda^2 U, 0 \leq t < +\infty, -\infty < \lambda < +\infty\\
                    &U(0, \lambda) = \ee^{-\ii \lambda \xi},\, U_t(0, \lambda) = 0
                \end{aligned}
            \right.
        \end{equation*}
        解之,得到
        \[U(t, \lambda) = C_1(\lambda) \ee^{\ii a \lambda t} + C_2(\lambda) \ee^{-\ii a \lambda t}\]
        根据初始条件得到
        \[U(t, \lambda) = \frac{1}{2}\left(\ee^{\ii \lambda \left(-\xi + a t\right)} + \ee^{\ii \lambda \left(-\xi - a t\right)}\right)\]
        反变换就得到
        \[u(t, x) = \frac{1}{2}\left(\delta(x - \xi - a t) + \delta(x - \xi + a t)\right)\]
    \end{solution}
\end{problembox}

一般来说,对于热方程和波动方程,方程本身总具有形式$u_t = Lu + f(t, M)$或者$u_{tt} = Lu + f(t, M)$,
这里的$L$是一个空间线性微分算子,以一维情况为例,就是$1$(认为是$0$阶导数)、$\pdv{}{x}$、$\pdv[2]{}{x}$等等的线性组合,$M$用于标记位置坐标。
然而,通过前面的学习我们发现,即便对于同一个方程本身而言,不同的边界条件也会给出不同的解。
基本解法的思想是,对同一个方程,用统一的方式去处理不同的边界条件,从而简化思考的流程。下面我们将直接给出关于基本解的定义和定理:
\begin{theorem}{热方程的基本解}{thm:3.1}
    称初值问题
    \begin{equation*}
        \left\{
            \begin{aligned}
                &U_{t} = LU, 0 \leq t < +\infty, M \in \mathbb{R}^n\\
                &U(0, M) = \delta(M)
            \end{aligned}
        \right.
    \end{equation*}
    的解$U(t, M)$为方程$u_t = Lu + f(t, M)$的初值问题(也称作柯西问题)的基本解。则方程
    \begin{equation*}
        \left\{
            \begin{aligned}
                &u_t = Lu + f(t, M), 0 \leq t < +\infty, M \in \mathbb{R}^n\\
                &U(0, M) = \varphi(M)
            \end{aligned}
        \right.
    \end{equation*}
    的解为
    \[u(t, M) = U(t, M) * \varphi(M) + \int_{0}^{t} f(\tau, M) * U(t - \tau, M) \dd{\tau}\]
    这里卷积的定义为
    \[f(t_1, M) * g(t_2, M) = \int_{\text{全空间}} f(t_1, M - M_0) g(t_2, M_0) \dd{M_0}\]
\end{theorem}
这个公式在直观上也不难理解:
\[\text{方程的解} = \text{基本解和初始状态的卷积} + \text{非齐次和基本解的卷积}\]
\begin{problembox}
    \begin{example}
        求方程$u_t = a^2 u_{xx} + bu$的柯西问题的基本解,$a>0$。
    \end{example}
    \begin{solution}
        即求解定解问题:
        \begin{equation*}
            \left\{
                \begin{aligned}
                    &u_t = a^2 u_{xx} + b u, 0 \leq t < +\infty, -\infty < x < +\infty\\
                    &u(0, x) = \delta(x)
                \end{aligned}
            \right.
        \end{equation*}
    做傅里叶变换。记
    \[U(t, \lambda) = \int_{-\infty}^{+\infty} u(t, x) \ee^{\ii \lambda x} \dd{x}\]
    则有
    \begin{equation*}
        \left\{
            \begin{aligned}
                &U_t = - a^2 \lambda^2 U + b U, 0 \leq t < +\infty, -\infty < \lambda < +\infty\\
                &U(0, \lambda) = 1
            \end{aligned}
        \right.
    \end{equation*}
    这个微分方程很好求解,答案是
    \[U(t, \lambda) = \ee^{-\left(a^2 \lambda^2 - b\right)t}\]
    最后就是反变换:
    \begin{align*}
        u(t, x) &= \frac{1}{2 \pi} \int_{-\infty}^{+\infty} \ee^{-\left(a^2 \lambda^2 - b\right)t} \ee^{\ii \lambda x} \dd{\lambda}\\
        &= \frac{1}{2 \pi } \int_{-\infty}^{+\infty} \ee^{-\left(\lambda a \sqrt{t} + \frac{\ii x}{2 a \sqrt{t}}\right)^2 + b t - \frac{x^2}{4 a^2 t}} \dd{\lambda}\\
        &= \frac{\ee^{b t - \frac{x^2}{4 a^2 t}}}{2 \pi} \times \frac{1}{a\sqrt{t}} \int_{-\infty}^{+\infty} \ee^{-\left(\lambda a \sqrt{t} + \frac{\ii x}{2 a \sqrt{t}}\right)^2} \dd{\left(a \sqrt{t} \lambda\right)}\\
        &= \frac{\ee^{b t - \frac{x^2}{4 a^2 t}}}{2 \pi} \times \frac{1}{a\sqrt{t}} \int_{-\infty}^{+\infty} \ee^{-\lambda^2} \dd{\lambda}\\
        &= \frac{\ee^{b t - \frac{x^2}{4 a^2 t}}}{\sqrt{4 a^2 \pi t}}
    \end{align*}
    \end{solution}
\end{problembox}
\begin{problembox}
    \begin{exercise}
        求方程$u_t = a u_x$的柯西问题的基本解。
    \end{exercise}
    \begin{exercise}
        求二维热方程的柯西问题的基本解。
    \end{exercise}
\end{problembox}

下面我们给出波动方程的基本解的定义和有关定理:
\begin{theorem}{波动方程的基本解}{thm:3.2}
    称初值问题
    \begin{equation*}
        \left\{
            \begin{aligned}
                &U_{tt} = LU, 0 \leq t < +\infty, M \in \mathbb{R}^n\\
                &U(0, M) = 0,\, U_t(0, M) = \delta(M)
            \end{aligned}
        \right.
    \end{equation*}
    的解$U(t, M)$为方程$u_{tt} = Lu + f(t, M)$的初值问题的基本解。则方程
    \begin{equation*}
        \left\{
            \begin{aligned}
                &u_{tt} = Lu + f\left(t, M\right), 0 \leq t < +\infty, M \in \mathbb{R}^n\\
                &u(0, M) = \varphi(M),\, u_t(0, M) = \psi(M)
            \end{aligned}
        \right.
    \end{equation*}
    的解为
    \[u(t, M) = \pdv{}{t}\left[U(t, M) * \varphi(M)\right] + U(t, M) * \psi(M) + \int_{0}^{t} U(t - \tau, M) * f(\tau, M) \dd{\tau}\]
\end{theorem}
我们发现最后的公式和热方程的很像,只是要考虑两个初始条件的贡献。
\begin{problembox}
    \begin{example}
        求三维波动方程的基本解,即求解方程:
        \begin{equation*}
            \left\{
                \begin{aligned}
                    &U_{tt} = a^3 \Delta_3 U, 0 \leq t < +\infty, -\infty < x, y, z < +\infty\\
                    &U(0, x, y, z) = 0,\, U_t(0, x, y, z) = \delta(x, y, z)
                \end{aligned}
            \right.
        \end{equation*}
    \end{example}
    \begin{solution}
        做三维的傅里叶变换
        \[\bar{U}(t, \lambda, \mu, \nu) = \int_{\mathbb{R}^3} u(t, x, y, z) \ee^{-\ii \left(\lambda x + \mu y + \nu z\right)} \dd{x} \dd{y} \dd{z}\]
        则有
        \begin{equation*}
            \left\{
                \begin{aligned}
                    &\bar{U}_{tt} = - a^2 \left(\lambda^2 + \mu^2 + \nu^2\right) \bar{U}, 0 \leq t < +\infty, -\infty < \lambda, \mu, \nu < +\infty\\
                    &\bar{U}(0, \lambda, \mu, \nu) = 0,\, \bar{U}_t(0, \lambda, \mu, \nu) = 1
                \end{aligned}
            \right.
        \end{equation*}
        这个方程具有很好的球对称性,所以用球坐标
        \[\lambda = \rho \sin \theta \cos \phi,\, \mu = \rho \sin \theta \sin \phi,\, \nu = \rho \cos \theta\]
        则
        \begin{equation*}
            \left\{
                \begin{aligned}
                    &\bar{U}_{tt} = - a^2 \rho^2 \bar{U}, 0 \leq t < +\infty, -\infty < \lambda, \mu, \nu < +\infty\\
                    &\bar{U}(0, \lambda, \mu, \nu) = 0,\, \bar{U}_t(0, \lambda, \mu, \nu) = 1
                \end{aligned}
            \right.
        \end{equation*}
        考虑初始条件之后解之:
        \[\bar{U} = \frac{\sin(a \rho t)}{a \rho}\]
        最后的任务就是计算傅里叶反变换
        \[U = \frac{1}{\left(2 \pi\right)^3} \int_{\mathbb{R}^3} \frac{\sin(a \rho t)}{a \rho} \ee^{\ii\left(\lambda x + \mu y + \nu z\right)} \dd{x} \dd{y} \dd{z}\]
        因为球对称性,所以一定有$U = U(t, r)$。那么$\lambda x + \mu y + \nu z = \rho r \cos \theta$。并且回忆球坐标的积分换元:$\dd{x} \dd{y} \dd{z} = \rho^2 \sin\theta \dd{\rho} \dd{\theta} \dd{\phi}$,最后我们的积分化为
        \begin{align*}
            U(t, r) &= \frac{1}{\left(2 \pi\right)^3} \int_{0}^{+\infty} \int_{0}^{2 \pi} \int_{0}^{\pi} \frac{\sin(a \rho t)}{a \rho} \ee^{\ii \rho r \cos \theta} \rho^2 \sin\theta \dd{\rho} \dd{\theta} \dd{\phi}\\
            &= \frac{1}{\left(2 \pi\right)^2} \int_{0}^{+\infty} \int_{0}^{2 \pi} \ee^{\ii \rho r \cos \theta} \rho^2 \sin\theta \dd{\rho} \dd{\theta}\\
            &= \frac{1}{\left(2 \pi\right)^2} \int_{0}^{+\infty} \frac{\sin(a \rho t)}{\ii a r} \ee^{\ii \rho r \cos \theta}|_{0}^{\pi} \dd{\rho}\\
            &= \frac{1}{\left(2 \pi\right)^2} \int_{0}^{+\infty} \frac{2 \sin(\rho r) \sin(a \rho t)}{a r} \dd{\rho}\\
            &= \frac{1}{8 \pi^2 a r} \int_{-\infty}^{+\infty} \left[\cos(\rho r - a \rho t) - \cos(\rho r  + a \rho t)\right] \dd{\rho}\\
            &= \frac{1}{4 \pi a r}\left[\delta(r - at) -\delta(r + at)\right]
        \end{align*}
        由于$r, a, t > 0$,故有
        \[U(t, r) = \frac{\delta(r - at)}{4 \pi a r}\]
    \end{solution}
\end{problembox}
\begin{problembox}
    \begin{exercise}
        求一维波动方程的基本解。
    \end{exercise}
    \begin{exercise}
        求二维波动方程的基本解。
    \end{exercise}
\end{problembox}
\subsection{格林函数}
前面的基本解方法针对的是含时的热方程和波动方程\footnote{实际上对不含时间的情况也有基本解的定义,但此时基本解方法和格林函数法没有什么区别。}。对于不含时的泊松方程,我们也有类似的解决方案,即格林函数。
同样的,我们直接给出定义和定理:
\begin{theorem}{泊松公式}{thm:3.3}
    称定解问题
    \begin{equation*}
        \left\{
            \begin{aligned}
                &\laplacian_{M}{G(M;M_0)} = -\delta(M - M_0),\, M, M_0 \in \Omega,\\
                &G|_{M \in \partial \Omega} = 0
            \end{aligned}
        \right.
    \end{equation*}
    的解为泊松方程第I边值问题的\textbf{格林函数},这里的$\Omega$可以是一维、二维或者三维的区域。则方程
    \begin{equation*}
        \left\{
            \begin{aligned}
                &\laplacian{u} = -f(M),\, M \in \Omega,\\
                &u|_{\partial \Omega} = \varphi(M)
            \end{aligned}
        \right.
    \end{equation*}
    的解为
    \[u(M) = \int_{V} f(M_0)G(M;M_0) \dd{M_0} - \oint_{\partial V} \varphi(M_0) \pdv{_{M_0}G(M;M_0)}{\vb{n}} \dd{D_0}\]
    这里$\dd{M_0}$是区域$\Omega$的“体积元”,$\dd{D_0}$是边界$\partial \Omega$的“面积元”。
\end{theorem}