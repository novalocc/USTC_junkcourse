\chapter{偏微分方程的基本概念和简单方法}
\section{这门课里我们要解什么方程?}
在这门课中,因为会大量的出现求导数的操作,所以我们规定:\textcolor{red}{一个下标表示对该变量求一次导数}。例如$u_t = \pdv{u}{t}, u_{xx} = \pdv[2]{u}{x}$。
同时,物理中的函数性质都很好,所以我们认为$u$想怎么导就怎么导。
整个课程里我们只会碰到三类方程\footnote{对于拉普拉斯算符,这里为了统一不同维度的写法,用的是$\laplacian{}$。按照这门课的记号,以2维情况为例,要写作$\Delta_2$。}:
\begin{itemize}
    \item 波动方程:$u_{tt} = a^2 \laplacian{u} + f(t, \vb{x})$
    \item 热传导方程:$u_{t} = a^2 \laplacian{u} + f(t, \vb{x})$
    \item 泊松方程:$\laplacian{u} = f(t, \vb{x})$
\end{itemize}
其中$u = t(t, \vb{x})$(对于泊松方程,$u = u(\vb{x})$)是待求的函数。
特别地,如果上面的$f(t, \vb{x}) = 0$,我们称方程是\textbf{齐次的}。

\section{定解条件}
地球人都知道,对于一个微分方程,只有方程本身还不能把解确定下来。我们还需要一些限制条件使得方程有唯一解,这就是定解条件。定解条件有两类:
\begin{enumerate}
    \item 初始条件:$t = 0$时系统的状态。
    \item 边界条件:系统的边界状态,可以和时间有关。具体地说,有如下三类:
        \begin{itemize}
            \item I类(Dirichlet条件),即系统在边界的取值,如$u(t, 0)$等;
            \item II类(Neumann条件),即系统在边界处法向导数的取值,如$\pdv{u}{\vb{n}}$等;
            \item III类(Robin条件),是上面两种条件的线性组合。
            \item 自然边界条件,比方说温度不能是无穷大;周期边界条件,在柱坐标和球坐标中很常见\footnote{我们后面会见到这两个边界条件的例子。}。
        \end{itemize}
\end{enumerate}

定解条件总是一些等式。对于I、II和III类条件,我们可以先通过移项,使一边只有$u$和它的导数。如果另一边是$0$,我们就称该条件是\textbf{齐次的},否则就是非齐次的。
\begin{problembox}
    \begin{example}
        边界条件$u(t, 0) = 0,\, u_x(t, 0) = 0$是齐次的边界条件。
    \end{example}
    \begin{example}
        初始条件$u(0, x) = \varphi(x),\, u_t(0, x) = \psi(x)$是非齐次的边界条件。
    \end{example}
\end{problembox}

\section{线性偏微分方程}
\subsection{一阶线性偏微分方程}
这类方程具有形式
\[\sum_{i}^{n} b_i(x_1, x_2, \dots, x_n) \pdv{u}{x_i} = f(x_1, x_2, \dots, x_n)\]
求解的方法是:
\begin{enumerate}
    \item 写出\color{red}特征方程
        \[\frac{\dd{x_1}}{b_1} = \frac{\dd{x_2}}{b_2} = \cdots = \frac{\dd{x_n}}{b_n}\]
        \color{black}
    \item 根据特征方程,解出$n - 1$个不同的全微分$\dd{\varphi_i} = 0$。设$\xi_i = \varphi_i$,然后补上一个$\xi_n$使得新的变量$\left\{\xi_i\right\}$是相互独立的(即雅可比行列式不为0,但其实根本不需要计算雅可比,一眼就能补出来一个)
    \item 用新的$\left\{\xi_i\right\}$去换元,求解方程。
\end{enumerate}
\begin{problembox}
    \begin{example}
        设$u = u(x, y, z)$,求$u_x + \ee^y u_y = \ee^{-y}$的通解。
    \end{example}
    \begin{solution}
        特征方程
        \[\frac{\dd{x}}{1} = \frac{\dd{y}}{\ee^y} = \frac{\dd{z}}{0}\]
        找两个全微分,一个是显然的
        \[\dd{z} = 0 \Rightarrow \xi_1 = z\]
        另一个是
        \[\frac{\dd{x}}{1} = \frac{\dd{y}}{\ee^y} \Rightarrow \dd{x + \ee^{-y}} = 0 \Rightarrow \xi_2 = x + \ee^{-y}\]
        最后补一个,显然补上
        \[\xi_3 = x\]
        是满足要求的。下面要做的就是用链式法则替换求导:
        \[\pdv{}{x} = \sum_{i} \pdv{\xi_i}{x}\pdv{}{\xi_i} = \pdv{}{\xi_2} + \pdv{}{\xi_3}\]
        \[\pdv{}{y} = \sum_{i} \pdv{\xi_i}{y}\pdv{}{\xi_i} = -\ee^{-y}\pdv{}{\xi_2}\]
        \[\pdv{}{z} = \sum_{i} \pdv{\xi_i}{z}\pdv{}{\xi_i} = \pdv{}{\xi_1}\]
        代入原方程,得到
        \[u_{\xi_3} = \xi_2 - \xi_3\]
        这就可以直接积分了:
        \[u = f(\xi_1, \xi_2) + \int_{0}^{\xi_3}\left(\xi_2 - s\right) \dd{s} = f\left(\xi_1, \xi_2\right) + \xi_2 \xi_3 - \frac{\xi_3^2}{2} = f(z, x + \ee^{-y}) + x \ee^{-y} + \frac{x^2}{2}\]
        其中$f$是任意一个二元函数。
    \end{solution}
\end{problembox}
\begin{problembox}
    \begin{exercise}
        设$u = u\left(t, x\right)$,求解以下问题:
        \begin{equation*}
            \left\{
                \begin{aligned}
                    &u_t = a u_x,\, t >0,\, -\infty < x < +\infty, \\
                    &u(0, x) = \sin x.
                \end{aligned}
            \right.
        \end{equation*}
    \end{exercise}
\end{problembox}
\subsection{二阶线性偏微分方程}
这种方程的形式是
\[A(x, y) u_{xx} + 2 B(x, y) u_{xy} + C(x, y) u_{yy} + D(x, y) u_x + E(x, y) u_y + F(x, y) u = 0\]
解法是
\begin{enumerate}
    \item 写出\color{red}特征方程
        \[\frac{A}{\left(\dd{x}\right)^2} - \frac{2B}{\dd{x}\dd{y}} + \frac{C}{\left(\dd{y}\right)^2} = 0\]
        \color{black}
    \item 求解特征方程,有三种情况:
        \begin{itemize}
            \item 双曲型,即有两个实的全微分$\dd{\varphi} = 0,\, \dd{\psi} = 0$,作换元$\xi = \varphi,\, \eta = \psi$。得到的方程往往具有形式$u_{\xi \eta} = \dots$。
            \item 椭圆型,即有两个复的全微分$\dd{(\varphi + \ii \psi)} = 0,\, \dd{(\varphi - \ii \psi)} = 0$,作换元$\xi = \varphi,\, \eta = \psi$。得到的方程往往具有形式$u_{\xi \xi} + u_{\eta \eta} = \dots$。
            \item 双曲型,即只有一个实的全微分$\dd{\varphi} = 0$,作换元$\xi = \varphi$,然后补一个$\eta = \psi$(当然要独立,即雅可比不为0)。得到的方程往往具有形式$u_{\xi \xi} = \dots$。
        \end{itemize}
\end{enumerate}
\begin{problembox}
    \begin{example}
        设$u = u(x, y)$,求$u_{xx} + 3u_{xy} + 2u_{yy} = 0$的通解。
    \end{example}
    \begin{solution}
        特征方程
        \[\frac{1}{\left(\dd{x}\right)^2} - \frac{3}{\dd{x}\dd{y}} + \frac{2}{\left(\dd{y}\right)^2} = 0\]
        因式分解一下:
        \[\left(\frac{1}{\dd{x}} - \frac{1}{\dd{y}}\right)\left(\frac{1}{\dd{x}} - \frac{2}{\dd{y}}\right) = 0\]
        这就得到了
        \[\xi = x - y,\, \eta = 2x - y\]
        然后作导数的替换。先算一阶导:
        \[\pdv{}{x} = \pdv{}{\xi} + 2\pdv{}{\eta},\, \pdv{}{y} = -\pdv{}{\xi} - \pdv{}{\eta}\]
        这样可以帮助我们很快算出二阶导:
        \[\pdv[2]{}{x} = \left(\pdv{}{\xi} + 2\pdv{}{\eta}\right)^2 = \pdv[2]{}{\xi} + 4\pdv{}{\xi}{\eta} + 4\pdv[2]{}{\eta}\]
        \[\pdv{}{x}{y} = \left(\pdv{}{\xi} + 2\pdv{}{\eta}\right)\left(-\pdv{}{\xi} - \pdv{}{\eta}\right) = -\pdv[2]{}{\xi} - 3\pdv{}{\xi}{\eta} - 2 \pdv[2]{}{\eta}\]
        \[\pdv[2]{}{y} = \left(-\pdv{}{\xi} - \pdv{}{\eta}\right)^2 = \pdv[2]{}{\xi} + 2 \pdv{}{\xi}{\eta} + \pdv[2]{}{\eta}\]   
        代入原方程,得到
        \[u_{\xi \eta} = \xi - \eta \Rightarrow u_\xi = \xi \eta - \frac{\eta^2}{2} + f\left(\xi\right)\]
        于是
        \[u = \frac{\xi^2}{2} - \frac{\xi \eta^2}{2} + f(\xi) + g(\eta) = \frac{3 x^2 y - 2 x^3 - x y^2}{2} + f(x - y) + g(2x - y)\]     
        其中$f$和$g$是任意一个一元函数。
    \end{solution}
\end{problembox}
\begin{problembox}
    \begin{exercise}
        设$u = u(x, y)$,求$u_{xx} + 2 u_{xy} + u_{yy} = 0$的通解。
    \end{exercise}
\end{problembox}
\section{达朗贝尔公式}
达朗贝尔公式针对的是齐次的无界弦振动方程的初值问题,即
\begin{equation*}
    \left\{
        \begin{aligned}
            &u_{tt} = a^2 u_{xx},\, 0 \leq t < +\infty,\, -\infty < x < +\infty,\\
            &u(0, x) = \varphi(x),\, u_t(0, x) = \psi(x).
        \end{aligned}
    \right.
\end{equation*}
为此,需要先求$u_{tt} - a^2 u_{xx} = 0$的通解。这在上一小节已经学过,写出特征方程
\[\frac{1}{\left(\dd{x}\right)^2} - \frac{a^2}{\left(\dd{y}\right)^2} = 0 \Rightarrow \xi = x - a t,\, \eta = x + at\]
原方程就化为
\[u_{\xi \eta} = 0 \Rightarrow u = f(\xi) + g(\eta) = f(x - a t) + g(x + a t)\]
下一步是根据初始条件解出两个待定函数
\begin{equation*}
    \left\{
        \begin{aligned}
            &f(x) + g(x) = \varphi(x),\\
            &f'(x) - g'(x) = \frac{\psi(x)}{a}.
        \end{aligned}
    \right.
\end{equation*}
解得
\begin{equation*}
    \left\{
        \begin{aligned}
            &f(x) = \frac{1}{2}\varphi(x) + \frac{1}{2a} \int_{0}^{x} \psi(s) \dd{s} + C,\\
            &g(x) = \frac{1}{2}\varphi(x) - \frac{1}{2a} \int_{0}^{x} \psi(s) \dd{s} - C.
        \end{aligned}
    \right.
\end{equation*}
代入就得到\color{red}达朗贝尔公式
\[u(t, x) = \frac{\varphi(x + a t) + \varphi(x - a t)}{2} + \frac{1}{2a}\int_{x - a t}^{x + a t}\psi(s)\dd{s}\]
\color{black}
然而实际常常遇见的是非齐次的弦振动方程:
\begin{equation*}
    \left\{
        \begin{aligned}
            &u_{tt} = a^2 u_{xx} + f(t, x),\, 0 \leq t < +\infty,\, -\infty < x < +\infty,\\
            &u(0, x) = \varphi(x),\, u_t(0, x) = \psi(x).
        \end{aligned}
    \right.
\end{equation*}
解题方法如下:
\begin{itemize}
    \item 先找方程$v_{tt} = a^2 v_{xx} + f(t, x)$的一个特解$v(t, x)$\footnote{关于如何找特解,请大家自己询问一下AI,或者翻一下数分B2的教材。};
    \item 然后用$\tilde{u} = u - v$建立新的初值问题
        \begin{equation*}
            \left\{
                \begin{aligned}
                    &\tilde{u}_{tt} = a^2 \tilde{u}_{xx},\, 0 \leq t < +\infty,\, -\infty < x < +\infty,\\
                    &\tilde{u}(0, x) = \varphi(x) - v(0, x),\, \tilde{u}_t(0, x) \psi(x) - v_t(0, x).
                \end{aligned}
            \right.
        \end{equation*}
    \item 最后用达朗贝尔公式解出$\tilde{u}$,就得到$u = \tilde{u} + v$。
\end{itemize}
在考试中我们会用更快的一种分解的方法,见下一小节。尽管如此,练习上面的步骤仍然是有必要的。
\begin{problembox}
    \begin{example}\label{eg:1.5}
        求解以下非齐次定解问题:
        \begin{equation*}
            \left\{
                \begin{aligned}
                    &u_{tt} = u_{xx} + f(t, x),\, 0 \leq t < +\infty,\, -\infty < x < +\infty,\\
                    &u(0, x) = x,\, u_t(0, x) = \sin x
                \end{aligned}
            \right.
        \end{equation*}
        其中$f(t, x) = x^2 \ee^{-t}$。
    \end{example}
    \begin{solution}
        先得找一个特解$v$。找特解的一般方法是,$f$长什么样,我们就猜特解长什么样。所以对于本题,我们设
        \[v(t, x) = \left(a^2 + b x + c\right)\ee^{-t}\]
        代入原方程然后比对一下系数,得到
        \[v(t, x) = \left(x^2 + 2\right) \ee^{-t}\]
        令$\tilde{u} = u - v$,得到
        \begin{equation*}
            \left\{
                \begin{aligned}
                    &\tilde{u}_{tt} = \tilde{u}_{xx},\, 0 \leq t < +\infty,\, -\infty < x < +\infty,\\
                    &\tilde{u}(0, x) = x - x^2 - 2,\, \tilde{u}_t(0, x) = \sin x + x^2 + 2
                \end{aligned}
            \right.
        \end{equation*}
        下面就是愉快的套公式环节:
        \[\tilde{u}(t, x) = \frac{\left(x + t\right) - \left(x + t\right)^2 - 2 + \left(x - t\right) - \left(x - t\right)^2 - 2}{2} + \frac{1}{2}\int_{x - t}^{x + t} \left(\sin s + s^2 + 2\right) \dd{s}\]
        完成计算,结果是
        \[\tilde{u}(t, x) = -x^2 + x - 2 - t^2 + \sin x \sin t + x^2 t + \frac{t^3}{3} + 2 t\]
        从而
        \[u(t, x) = \tilde{u} + v = -x^2 + x - 2 - t^2 + \sin x \sin t + x^2 t + \frac{t^3}{3} + 2 t + \left(x^2 + 2\right) \ee^{-t}\]
    \end{solution}
\end{problembox}
\begin{problembox}
    \begin{exercise}
        把例\ref{eg:1.5}的$f(t, x)$换成$x \cos 2t$,重新解方程。
    \end{exercise}
    \begin{exercise}
        把例\ref{eg:1.5}的$f(t, x)$换成$\sin^2 x$,重新解方程。
    \end{exercise}
\end{problembox}
\section{叠加原理和冲量原理}