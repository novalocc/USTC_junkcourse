\chapter*{前言}
\addcontentsline{toc}{chapter}{前言}
数理方程A作为公认的石课,我认为这些知识应该在具体的课程中结合物理实例进行学习,而不是单独在一门课一股脑全部学完。
花费时间大量练习过一周就会完全忘记的知识点毫无意义,期末考试更是在筛选人形计算器。
而这门课程的作业,我更推荐用AI模型(比方说DeepSeek)去做,他们做得又快又好。

经过我以及诸多学长、同学的实践,在期末考试前三天乃至前一晚速通这门课是完全可行的。
本资料就是为了满足考试的需要,结合部分讲义和我自己的复习经验,帮助大家在尽可能短的时间内获得相对不错的分数。

如果你平常就认真学习数理方程这门课(我对认真学习一门课的定义是考前一周就开始学习),那这份资料对你的帮助可能不太大,我更推荐去系统学习田涌波老师或许雷叶老师的讲义。
如果你平时不是那么认真,只需要花一晚上跟着这份资料学一遍,知识点上就应该没有空白区。然后再花一天练习一下往年试卷还有助教的习题课,期末考试应该就没有本质上的困难。
资料中的例题都尽可能的附上了详细的解答,但是练习题我实在是懒得再敲了,就没有答案。如果需要练习的参考答案,请麻烦大家用AI做一下。

在写作这篇资料时,我主要参考了许雷叶老师的讲义,以及Google Gemini 2.5 Pro给出的若干建议,在此表达真挚的感谢。祝大家食用愉快!

\begin{itemize}
    \item 2025.6.24 本资料正式开始编写,目前版本号为v0.0.1。
\end{itemize}